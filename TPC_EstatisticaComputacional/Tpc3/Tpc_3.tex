% Options for packages loaded elsewhere
\PassOptionsToPackage{unicode}{hyperref}
\PassOptionsToPackage{hyphens}{url}
%
\documentclass[
]{article}
\usepackage{amsmath,amssymb}
\usepackage{lmodern}
\usepackage{iftex}
\ifPDFTeX
  \usepackage[T1]{fontenc}
  \usepackage[utf8]{inputenc}
  \usepackage{textcomp} % provide euro and other symbols
\else % if luatex or xetex
  \usepackage{unicode-math}
  \defaultfontfeatures{Scale=MatchLowercase}
  \defaultfontfeatures[\rmfamily]{Ligatures=TeX,Scale=1}
\fi
% Use upquote if available, for straight quotes in verbatim environments
\IfFileExists{upquote.sty}{\usepackage{upquote}}{}
\IfFileExists{microtype.sty}{% use microtype if available
  \usepackage[]{microtype}
  \UseMicrotypeSet[protrusion]{basicmath} % disable protrusion for tt fonts
}{}
\makeatletter
\@ifundefined{KOMAClassName}{% if non-KOMA class
  \IfFileExists{parskip.sty}{%
    \usepackage{parskip}
  }{% else
    \setlength{\parindent}{0pt}
    \setlength{\parskip}{6pt plus 2pt minus 1pt}}
}{% if KOMA class
  \KOMAoptions{parskip=half}}
\makeatother
\usepackage{xcolor}
\usepackage[margin=1in]{geometry}
\usepackage{color}
\usepackage{fancyvrb}
\newcommand{\VerbBar}{|}
\newcommand{\VERB}{\Verb[commandchars=\\\{\}]}
\DefineVerbatimEnvironment{Highlighting}{Verbatim}{commandchars=\\\{\}}
% Add ',fontsize=\small' for more characters per line
\usepackage{framed}
\definecolor{shadecolor}{RGB}{248,248,248}
\newenvironment{Shaded}{\begin{snugshade}}{\end{snugshade}}
\newcommand{\AlertTok}[1]{\textcolor[rgb]{0.94,0.16,0.16}{#1}}
\newcommand{\AnnotationTok}[1]{\textcolor[rgb]{0.56,0.35,0.01}{\textbf{\textit{#1}}}}
\newcommand{\AttributeTok}[1]{\textcolor[rgb]{0.77,0.63,0.00}{#1}}
\newcommand{\BaseNTok}[1]{\textcolor[rgb]{0.00,0.00,0.81}{#1}}
\newcommand{\BuiltInTok}[1]{#1}
\newcommand{\CharTok}[1]{\textcolor[rgb]{0.31,0.60,0.02}{#1}}
\newcommand{\CommentTok}[1]{\textcolor[rgb]{0.56,0.35,0.01}{\textit{#1}}}
\newcommand{\CommentVarTok}[1]{\textcolor[rgb]{0.56,0.35,0.01}{\textbf{\textit{#1}}}}
\newcommand{\ConstantTok}[1]{\textcolor[rgb]{0.00,0.00,0.00}{#1}}
\newcommand{\ControlFlowTok}[1]{\textcolor[rgb]{0.13,0.29,0.53}{\textbf{#1}}}
\newcommand{\DataTypeTok}[1]{\textcolor[rgb]{0.13,0.29,0.53}{#1}}
\newcommand{\DecValTok}[1]{\textcolor[rgb]{0.00,0.00,0.81}{#1}}
\newcommand{\DocumentationTok}[1]{\textcolor[rgb]{0.56,0.35,0.01}{\textbf{\textit{#1}}}}
\newcommand{\ErrorTok}[1]{\textcolor[rgb]{0.64,0.00,0.00}{\textbf{#1}}}
\newcommand{\ExtensionTok}[1]{#1}
\newcommand{\FloatTok}[1]{\textcolor[rgb]{0.00,0.00,0.81}{#1}}
\newcommand{\FunctionTok}[1]{\textcolor[rgb]{0.00,0.00,0.00}{#1}}
\newcommand{\ImportTok}[1]{#1}
\newcommand{\InformationTok}[1]{\textcolor[rgb]{0.56,0.35,0.01}{\textbf{\textit{#1}}}}
\newcommand{\KeywordTok}[1]{\textcolor[rgb]{0.13,0.29,0.53}{\textbf{#1}}}
\newcommand{\NormalTok}[1]{#1}
\newcommand{\OperatorTok}[1]{\textcolor[rgb]{0.81,0.36,0.00}{\textbf{#1}}}
\newcommand{\OtherTok}[1]{\textcolor[rgb]{0.56,0.35,0.01}{#1}}
\newcommand{\PreprocessorTok}[1]{\textcolor[rgb]{0.56,0.35,0.01}{\textit{#1}}}
\newcommand{\RegionMarkerTok}[1]{#1}
\newcommand{\SpecialCharTok}[1]{\textcolor[rgb]{0.00,0.00,0.00}{#1}}
\newcommand{\SpecialStringTok}[1]{\textcolor[rgb]{0.31,0.60,0.02}{#1}}
\newcommand{\StringTok}[1]{\textcolor[rgb]{0.31,0.60,0.02}{#1}}
\newcommand{\VariableTok}[1]{\textcolor[rgb]{0.00,0.00,0.00}{#1}}
\newcommand{\VerbatimStringTok}[1]{\textcolor[rgb]{0.31,0.60,0.02}{#1}}
\newcommand{\WarningTok}[1]{\textcolor[rgb]{0.56,0.35,0.01}{\textbf{\textit{#1}}}}
\usepackage{graphicx}
\makeatletter
\def\maxwidth{\ifdim\Gin@nat@width>\linewidth\linewidth\else\Gin@nat@width\fi}
\def\maxheight{\ifdim\Gin@nat@height>\textheight\textheight\else\Gin@nat@height\fi}
\makeatother
% Scale images if necessary, so that they will not overflow the page
% margins by default, and it is still possible to overwrite the defaults
% using explicit options in \includegraphics[width, height, ...]{}
\setkeys{Gin}{width=\maxwidth,height=\maxheight,keepaspectratio}
% Set default figure placement to htbp
\makeatletter
\def\fps@figure{htbp}
\makeatother
\setlength{\emergencystretch}{3em} % prevent overfull lines
\providecommand{\tightlist}{%
  \setlength{\itemsep}{0pt}\setlength{\parskip}{0pt}}
\setcounter{secnumdepth}{-\maxdimen} % remove section numbering
\ifLuaTeX
  \usepackage{selnolig}  % disable illegal ligatures
\fi
\IfFileExists{bookmark.sty}{\usepackage{bookmark}}{\usepackage{hyperref}}
\IfFileExists{xurl.sty}{\usepackage{xurl}}{} % add URL line breaks if available
\urlstyle{same} % disable monospaced font for URLs
\hypersetup{
  pdftitle={Estatistica Computacional -\textgreater{} TPC 3},
  pdfauthor={Diogo Alexandre Alonso De Freitas},
  hidelinks,
  pdfcreator={LaTeX via pandoc}}

\title{Estatistica Computacional -\textgreater{} TPC 3}
\author{Diogo Alexandre Alonso De Freitas}
\date{2022-09-20}

\begin{document}
\maketitle

\hypertarget{experiuxeancia-lanuxe7amento-de-um-dado-de-6-faces-e-equilibrado-duas-vezes}{%
\subsubsection{Experiência: Lançamento de um dado de 6 faces e
equilibrado, duas
vezes}\label{experiuxeancia-lanuxe7amento-de-um-dado-de-6-faces-e-equilibrado-duas-vezes}}

\hypertarget{definiuxe7uxe3o-da-variuxe1vel-aleatuxf3ria}{%
\paragraph{Definição da variável
aleatória:}\label{definiuxe7uxe3o-da-variuxe1vel-aleatuxf3ria}}

\hypertarget{x---soma-do-nuxfamero-de-faces-voltados-para-cima-em-dois-lanuxe7amentos-de-um-cubo-numerado-de-1-a-6-equilibrada}{%
\subparagraph{X -\textgreater{} Soma do número de faces, voltados para
cima, em dois lançamentos de um cubo (numerado de 1 a 6)
equilibrada}\label{x---soma-do-nuxfamero-de-faces-voltados-para-cima-em-dois-lanuxe7amentos-de-um-cubo-numerado-de-1-a-6-equilibrada}}

\hypertarget{exercicio-1.}{%
\subsection{Exercicio 1.}\label{exercicio-1.}}

Tabela com as respetivas combinações e as respetivas probabilidades de
ocorrerem (Dado equilibrado)

\begin{Shaded}
\begin{Highlighting}[]
\NormalTok{dado }\OtherTok{\textless{}{-}} \DecValTok{1}\SpecialCharTok{:}\DecValTok{6}
\NormalTok{lancamento\_dados }\OtherTok{\textless{}{-}} \FunctionTok{expand.grid}\NormalTok{(}\StringTok{"lancamento 1"} \OtherTok{=}\NormalTok{ dado, }\StringTok{"lancamento 2"} \OtherTok{=}\NormalTok{ dado)}
\NormalTok{lancamento\_dados}\SpecialCharTok{$}\NormalTok{Prob }\OtherTok{\textless{}{-}} \FunctionTok{rep}\NormalTok{(}\DecValTok{1}\SpecialCharTok{/}\FunctionTok{nrow}\NormalTok{(lancamento\_dados), }\AttributeTok{times =} \FunctionTok{nrow}\NormalTok{(lancamento\_dados))}
\NormalTok{lancamento\_dados}
\end{Highlighting}
\end{Shaded}

É a mesma tabela que a anterior, mas possui uma coluna a mais; tendo
esta a soma das 2 faces que sairam (Para assim desta forma conseguir
criar a função de probabilidade)

\begin{Shaded}
\begin{Highlighting}[]
\NormalTok{lancamento\_dados}\SpecialCharTok{$}\NormalTok{Soma\_Faces }\OtherTok{\textless{}{-}} \FunctionTok{rowSums}\NormalTok{(lancamento\_dados[,}\DecValTok{1}\SpecialCharTok{:}\DecValTok{2}\NormalTok{])}
\NormalTok{lancamento\_dados}
\end{Highlighting}
\end{Shaded}

Na tabela abaixo, é possivel verificar que \(X\) contém os valores da
variável, nomeadamente, o \emph{Soma\_Faces};

\emph{Prob} contém a respetiva função de probabilidade.

\begin{Shaded}
\begin{Highlighting}[]
\NormalTok{X }\OtherTok{\textless{}{-}} \FunctionTok{aggregate}\NormalTok{(Prob}\SpecialCharTok{\textasciitilde{}}\NormalTok{Soma\_Faces, lancamento\_dados, sum)}
\NormalTok{X}
\end{Highlighting}
\end{Shaded}

\hypertarget{exercicio-2.}{%
\subsection{Exercicio 2.}\label{exercicio-2.}}

\includegraphics{Tpc_3_files/figure-latex/unnamed-chunk-5-1.pdf}

\hypertarget{exercicio-3.}{%
\subsection{Exercicio 3.}\label{exercicio-3.}}

\hypertarget{definiuxe7uxe3o-da-funuxe7uxe3o-de-distribuiuxe7uxe3o}{%
\subparagraph{Definição da função de
distribuição}\label{definiuxe7uxe3o-da-funuxe7uxe3o-de-distribuiuxe7uxe3o}}

A função de distribuição \(F(x)\) (notem a utilização de Maiúscula) para
uma variável aleatória discreta é uma função definida em
\textbf{patamares} (ou função em escada).

Podemos começar por definir os valores dos diferentes patamares, que
correspondem às probabilidades acumuladas em cada um dos pontos do
suporte de X.

Para isso podemos usar a função genérica \textbf{cumsum()}

\begin{Shaded}
\begin{Highlighting}[]
\NormalTok{FX\_pontos }\OtherTok{\textless{}{-}} \FunctionTok{cumsum}\NormalTok{(X}\SpecialCharTok{$}\NormalTok{Prob)}
\NormalTok{FX\_pontos}
\end{Highlighting}
\end{Shaded}

\begin{verbatim}
##  [1] 0.02777778 0.08333333 0.16666667 0.27777778 0.41666667 0.58333333
##  [7] 0.72222222 0.83333333 0.91666667 0.97222222 1.00000000
\end{verbatim}

\hypertarget{exercicio-4}{%
\subsection{Exercicio 4}\label{exercicio-4}}

\hypertarget{funuxe7uxe3o-de-distribuiuxe7uxe3o}{%
\subparagraph{Função de
distribuição}\label{funuxe7uxe3o-de-distribuiuxe7uxe3o}}

\[ F(x)=   \left \{
            \begin{array}{ll}
                  0  & x < 2 \\
                  0.02777778  & 2 \le x < 3 \\
                  0.08333333 & 3 \le x < 4 \\
                  0.16666667 & 4 \le x < 5 \\
                  0.27777778  & 5 \le x < 6 \\
                  0.41666667 & 6 \le x < 7 \\
                  0.58333333 & 7 \le x < 8 \\
                  0.72222222  & 8 \le x < 9 \\
                  0.83333333 & 9 \le x < 10 \\
                  0.91666667 & 10 \le x < 11 \\
                  0.97222222 & 11 \le x < 12 \\
                  1 & x \le 12 \\
\end{array} 
\right.  \]

\hypertarget{exercicio-5.}{%
\subsection{Exercicio 5.}\label{exercicio-5.}}

\hypertarget{representauxe7uxe3o-gruxe1fica-da-funuxe7uxe3o-de-distribuiuxe7uxe3o}{%
\subparagraph{Representação gráfica da função de
distribuição}\label{representauxe7uxe3o-gruxe1fica-da-funuxe7uxe3o-de-distribuiuxe7uxe3o}}

Para representar graficamente a função em todo o seu domínio vamos
recorrer a \textbf{stepfun}

\includegraphics{Tpc_3_files/figure-latex/unnamed-chunk-7-1.pdf}

\hypertarget{exercicio-6.}{%
\subsection{Exercicio 6.}\label{exercicio-6.}}

\hypertarget{recurso-uxe0-funuxe7uxe3o-de-probabilidade}{%
\paragraph{1 - Recurso à Função de
probabilidade}\label{recurso-uxe0-funuxe7uxe3o-de-probabilidade}}

Para obter a probabilidade de obter mais de 2 pontos no lançamento de um
dado, \(P[7 <= X <= 10]\), basta somar a função de probabilidade para os
valores da variável que respeitam a condição.

\begin{Shaded}
\begin{Highlighting}[]
\FunctionTok{sum}\NormalTok{(X[X}\SpecialCharTok{$}\NormalTok{Soma\_Faces }\SpecialCharTok{\textless{}=} \DecValTok{10} \SpecialCharTok{\&}\NormalTok{ X}\SpecialCharTok{$}\NormalTok{Soma\_Faces }\SpecialCharTok{\textgreater{}=} \DecValTok{7}\NormalTok{, ]}\SpecialCharTok{$}\NormalTok{Prob )}
\end{Highlighting}
\end{Shaded}

\begin{verbatim}
## [1] 0.5
\end{verbatim}

\hypertarget{recurso-uxe0-funuxe7uxe3o-de-distribuiuxe7uxe3o}{%
\paragraph{2 - Recurso à função de
distribuição}\label{recurso-uxe0-funuxe7uxe3o-de-distribuiuxe7uxe3o}}

Recordar que \[P[x_k<X \le x_t ]=\sum_{k+1}^t f(x_i)=F(x_t)-F(x_k)\]

Se pretendermos \(P[7\le X \le 10]\) basta calcular
\(P[X \le 9]-P[X < 5]=F(9)-F(5)=\)

\begin{verbatim}
## [1] 0.5
\end{verbatim}

\hypertarget{exercicio-7.}{%
\subsection{Exercicio 7.}\label{exercicio-7.}}

\hypertarget{recurso-uxe0-funuxe7uxe3o-de-probabilidade-1}{%
\paragraph{1 - Recurso à função de
probabilidade}\label{recurso-uxe0-funuxe7uxe3o-de-probabilidade-1}}

Para obter a probabilidade de obter mais de 2 pontos no lançamento de um
dado, \(P[7 < X <= 10]\), basta somar a função de probabilidade para os
valores da variável que respeitam a condição.

\begin{Shaded}
\begin{Highlighting}[]
\FunctionTok{sum}\NormalTok{(X[X}\SpecialCharTok{$}\NormalTok{Soma\_Faces }\SpecialCharTok{\textless{}=} \DecValTok{10} \SpecialCharTok{\&}\NormalTok{ X}\SpecialCharTok{$}\NormalTok{Soma\_Faces }\SpecialCharTok{\textgreater{}} \DecValTok{7}\NormalTok{, ]}\SpecialCharTok{$}\NormalTok{Prob )}
\end{Highlighting}
\end{Shaded}

\begin{verbatim}
## [1] 0.3333333
\end{verbatim}

\hypertarget{recurso-uxe0-funuxe7uxe3o-de-distribuiuxe7uxe3o-1}{%
\paragraph{2 - Recurso à função de
distribuição}\label{recurso-uxe0-funuxe7uxe3o-de-distribuiuxe7uxe3o-1}}

Se pretendermos \(P[7< X \le 10]\) basta calcular
\(P[X \le 9]-P[X < 6]=F(9)-F(6)=\)

\begin{verbatim}
## [1] 0.3333333
\end{verbatim}

\hypertarget{exercicio-8.}{%
\subsection{Exercicio 8.}\label{exercicio-8.}}

\hypertarget{recurso-uxe0-funuxe7uxe3o-de-probabilidade-2}{%
\paragraph{1 - Recurso à função de
probabilidade}\label{recurso-uxe0-funuxe7uxe3o-de-probabilidade-2}}

Para obter a probabilidade de obter mais de 2 pontos no lançamento de um
dado, \(P[7 < X < 10]\), basta somar a função de probabilidade para os
valores da variável que respeitam a condição.

\begin{Shaded}
\begin{Highlighting}[]
\FunctionTok{sum}\NormalTok{(X[X}\SpecialCharTok{$}\NormalTok{Soma\_Faces }\SpecialCharTok{\textless{}} \DecValTok{10} \SpecialCharTok{\&}\NormalTok{ X}\SpecialCharTok{$}\NormalTok{Soma\_Faces }\SpecialCharTok{\textgreater{}} \DecValTok{7}\NormalTok{, ]}\SpecialCharTok{$}\NormalTok{Prob )}
\end{Highlighting}
\end{Shaded}

\begin{verbatim}
## [1] 0.25
\end{verbatim}

\hypertarget{recurso-uxe0-funuxe7uxe3o-de-distribuiuxe7uxe3o-2}{%
\paragraph{2 - Recurso à função de
distribuição}\label{recurso-uxe0-funuxe7uxe3o-de-distribuiuxe7uxe3o-2}}

Se pretendermos \(P[7 < X < 10]\) basta calcular
\(P[X < 10]-P[X < 7]=F(10)-F(7)=\)

\begin{verbatim}
## [1] 0.25
\end{verbatim}

\hypertarget{exercicio-9.-valor-esperado-e-variuxe2ncia-numa-v.a.-discreta}{%
\subsection{Exercicio 9. (Valor esperado e variância numa v.a.
discreta)}\label{exercicio-9.-valor-esperado-e-variuxe2ncia-numa-v.a.-discreta}}

\hypertarget{muxe9dia}{%
\paragraph{1. Média}\label{muxe9dia}}

Visionando a fórmula \[E[X]=\sum_{1}^{n} x_i f(x_i)\]

Basta calcular a soma dos produtos da primeira com a segunda colunas
definidas:

\begin{Shaded}
\begin{Highlighting}[]
\NormalTok{miu\_X }\OtherTok{\textless{}{-}} \FunctionTok{sum}\NormalTok{(X}\SpecialCharTok{$}\NormalTok{Soma\_Faces}\SpecialCharTok{*}\NormalTok{X}\SpecialCharTok{$}\NormalTok{Prob)}
\NormalTok{miu\_X}
\end{Highlighting}
\end{Shaded}

\begin{verbatim}
## [1] 7
\end{verbatim}

\hypertarget{variuxe2ncia}{%
\paragraph{2. Variância}\label{variuxe2ncia}}

Para a variância, podemos aplicar a fórmula simplificada (média dos
quadrados menos o quadrado da média) ou a definição formal (média dos
quadrados dos desvios face à média).

a) média dos quadrados menos o quadrado da média

\begin{Shaded}
\begin{Highlighting}[]
\NormalTok{var\_X}\OtherTok{\textless{}{-}} \FunctionTok{sum}\NormalTok{((X}\SpecialCharTok{$}\NormalTok{Soma\_Faces}\SpecialCharTok{\^{}}\DecValTok{2}\NormalTok{)}\SpecialCharTok{*}\NormalTok{X}\SpecialCharTok{$}\NormalTok{Prob)}\SpecialCharTok{{-}}\NormalTok{miu\_X}\SpecialCharTok{\^{}}\DecValTok{2}
\FunctionTok{round}\NormalTok{(var\_X,}\DecValTok{4}\NormalTok{)}
\end{Highlighting}
\end{Shaded}

\begin{verbatim}
## [1] 5.8333
\end{verbatim}

b) média dos quadrados dos desvios face à média

\begin{Shaded}
\begin{Highlighting}[]
\FunctionTok{round}\NormalTok{(}\FunctionTok{sum}\NormalTok{((X}\SpecialCharTok{$}\NormalTok{Soma\_Faces}\SpecialCharTok{{-}}\NormalTok{miu\_X)}\SpecialCharTok{\^{}}\DecValTok{2}\SpecialCharTok{*}\NormalTok{X}\SpecialCharTok{$}\NormalTok{Prob),}\DecValTok{4}\NormalTok{)}
\end{Highlighting}
\end{Shaded}

\begin{verbatim}
## [1] 5.8333
\end{verbatim}

\hypertarget{desvio-padruxe3o}{%
\paragraph{3. Desvio-Padrão}\label{desvio-padruxe3o}}

\begin{Shaded}
\begin{Highlighting}[]
\FunctionTok{round}\NormalTok{(}\FunctionTok{sqrt}\NormalTok{(var\_X),}\DecValTok{2}\NormalTok{)}
\end{Highlighting}
\end{Shaded}

\begin{verbatim}
## [1] 2.42
\end{verbatim}

E assim o desvio-padrão será \({\sigma}_{X}=\) 2.42

\end{document}
